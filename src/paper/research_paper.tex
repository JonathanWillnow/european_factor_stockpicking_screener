\documentclass[11pt, a4paper, leqno]{article}
\usepackage{a4wide}
\usepackage[T1]{fontenc}
\usepackage[utf8]{inputenc}
\usepackage{float, afterpage, rotating, graphicx}
\usepackage{epstopdf}
\usepackage{longtable, booktabs, tabularx}
\usepackage{fancyvrb, moreverb, relsize}
\usepackage{eurosym, calc}
% \usepackage{chngcntr}
\usepackage{amsmath, amssymb, amsfonts, amsthm, bm}
\usepackage{caption}
\usepackage{mdwlist}
\usepackage{xfrac}
\usepackage{setspace}
\usepackage{xcolor}
\usepackage{subcaption}
\usepackage{minibox}
% \usepackage{pdf14} % Enable for Manuscriptcentral -- can't handle pdf 1.5
% \usepackage{endfloat} % Enable to move tables / figures to the end. Useful for some submissions.


\usepackage[
    natbib=true,
    bibencoding=inputenc,
    bibstyle=authoryear-ibid,
    citestyle=authoryear-comp,
    maxcitenames=3,
    maxbibnames=10,
    useprefix=false,
    sortcites=true,
    backend=biber
]{biblatex}
\AtBeginDocument{\toggletrue{blx@useprefix}}
\AtBeginBibliography{\togglefalse{blx@useprefix}}
\setlength{\bibitemsep}{1.5ex}
\addbibresource{refs.bib}





\usepackage[unicode=true]{hyperref}
\hypersetup{
    colorlinks=true,
    linkcolor=black,
    anchorcolor=black,
    citecolor=black,
    filecolor=black,
    menucolor=black,
    runcolor=black,
    urlcolor=black
}


\widowpenalty=10000
\clubpenalty=10000

\setlength{\parskip}{1ex}
\setlength{\parindent}{0ex}
\setstretch{1.5}


\begin{document}

\title{European Factor Stockpicking and Screener\thanks{Jonathan Willnow, University Bonn. Email: \href{mailto:jona.willnow@hotmail.de}{\nolinkurl{jona [dot] willnow [at] hotmail [dot] de}}.}}

\author{Jonathan Willnow}

\date{
    {\bf Preliminary -- please do not quote}
    \\[1ex]
    \today
}

\maketitle


\begin{abstract}
    The focus of this project is to build an interactive tool that helps individual investors in their buying decisions of individual stock. 
    The project tackles several aspects to lead to betetr ibnformed decisions: First, I deliberately omit the names of the stocks and report only the ticker symbol.
    This should help individuals to not bias themselves when tehy see an popular or unpopular stock since the decision should be purely based on the fundamentals.
    Secondly, apart from reporting standart metrics, I calculate the empirically proven Fama French Asset Pricing Factors. Thirdly, I come up with an on Score to score the stocks
    with respect to the asset pricing factors and reported metrics. This Score is not an ultimatively assasement since it does not capture all neccessary informations, 
    but investors can use it to get a first insight into a stock.
    This project is based on sraped stocks for Germany, Europe, North-America and Japan and collects the metrics used for assesment and calculation of the facotrs
    from finance.yahoo.com. The long term plan is to automate this project such that it serves as a reliable, automated way to screen stocks and improve stockpicking decisions.
\end{abstract}
\clearpage

\section{Introduction} % (fold)
\label{sec:introduction}

Stock market participation increased in the last years. While this is overall a positive develoment, many investors lack tools to base their investments on or are solely using their
gut feeling. While there are lots of tools available, I could not find a free tool that allows a comparison of stocks based on the Fama French Asset Pricing Factors - so I 
decided to build it myself. As this is work in progress, it is not available right now for the broad audience, but I shared the developed Dashboard with a big german speaking 
stock picking community, called "Kleine Finanzzeitung".



\section{Methodology}

\section{Code}
As this project was part of the course Effective Programming Practices for Economists by Prof. von Gaudecker, the corresponding repository 
that is available on github follows \citet{GaudeckerEconProjectTemplates}.

% section introduction (end)




\setstretch{1}
\printbibliography
\setstretch{1.5}




% \appendix

% The chngctr package is needed for the following lines.
% \counterwithin{table}{section}
% \counterwithin{figure}{section}

\end{document}
